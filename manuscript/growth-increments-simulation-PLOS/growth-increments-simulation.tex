% Template for PLoS
% Version 3.5 March 2018
%
% % % % % % % % % % % % % % % % % % % % % %
%
% -- IMPORTANT NOTE
%
% This template contains comments intended
% to minimize problems and delays during our production
% process. Please follow the template instructions
% whenever possible.
%
% % % % % % % % % % % % % % % % % % % % % % %
%
% Once your paper is accepted for publication,
% PLEASE REMOVE ALL TRACKED CHANGES in this file
% and leave only the final text of your manuscript.
% PLOS recommends the use of latexdiff to track changes during review, as this will help to maintain a clean tex file.
% Visit https://www.ctan.org/pkg/latexdiff?lang=en for info or contact us at latex@plos.org.
%
%
% There are no restrictions on package use within the LaTeX files except that
% no packages listed in the template may be deleted.
%
% Please do not include colors or graphics in the text.
%
% The manuscript LaTeX source should be contained within a single file (do not use \input, \externaldocument, or similar commands).
%
% % % % % % % % % % % % % % % % % % % % % % %
%
% -- FIGURES AND TABLES
%
% Please include tables/figure captions directly after the paragraph where they are first cited in the text.
%
% DO NOT INCLUDE GRAPHICS IN YOUR MANUSCRIPT
% - Figures should be uploaded separately from your manuscript file.
% - Figures generated using LaTeX should be extracted and removed from the PDF before submission.
% - Figures containing multiple panels/subfigures must be combined into one image file before submission.
% For figure citations, please use "Fig" instead of "Figure".
% See http://journals.plos.org/plosone/s/figures for PLOS figure guidelines.
%
% Tables should be cell-based and may not contain:
% - spacing/line breaks within cells to alter layout or alignment
% - do not nest tabular environments (no tabular environments within tabular environments)
% - no graphics or colored text (cell background color/shading OK)
% See http://journals.plos.org/plosone/s/tables for table guidelines.
%
% For tables that exceed the width of the text column, use the adjustwidth environment as illustrated in the example table in text below.
%
% % % % % % % % % % % % % % % % % % % % % % % %
%
% -- EQUATIONS, MATH SYMBOLS, SUBSCRIPTS, AND SUPERSCRIPTS
%
% IMPORTANT
% Below are a few tips to help format your equations and other special characters according to our specifications. For more tips to help reduce the possibility of formatting errors during conversion, please see our LaTeX guidelines at http://journals.plos.org/plosone/s/latex
%
% For inline equations, please be sure to include all portions of an equation in the math environment.
%
% Do not include text that is not math in the math environment.
%
% Please add line breaks to long display equations when possible in order to fit size of the column.
%
% For inline equations, please do not include punctuation (commas, etc) within the math environment unless this is part of the equation.
%
% When adding superscript or subscripts outside of brackets/braces, please group using {}.
%
% Do not use \cal for caligraphic font.  Instead, use \mathcal{}
%
% % % % % % % % % % % % % % % % % % % % % % % %
%
% Please contact latex@plos.org with any questions.
%
% % % % % % % % % % % % % % % % % % % % % % % %

\documentclass[10pt,letterpaper]{article}
\usepackage[top=0.85in,left=2.75in,footskip=0.75in]{geometry}

% amsmath and amssymb packages, useful for mathematical formulas and symbols
\usepackage{amsmath,amssymb}

% Use adjustwidth environment to exceed column width (see example table in text)
\usepackage{changepage}

% Use Unicode characters when possible
\usepackage[utf8x]{inputenc}

% textcomp package and marvosym package for additional characters
\usepackage{textcomp,marvosym}

% cite package, to clean up citations in the main text. Do not remove.
% \usepackage{cite}

% Use nameref to cite supporting information files (see Supporting Information section for more info)
\usepackage{nameref,hyperref}

% line numbers
\usepackage[right]{lineno}

% ligatures disabled
\usepackage{microtype}
\DisableLigatures[f]{encoding = *, family = * }

% color can be used to apply background shading to table cells only
\usepackage[table]{xcolor}

% array package and thick rules for tables
\usepackage{array}

% create "+" rule type for thick vertical lines
\newcolumntype{+}{!{\vrule width 2pt}}

% create \thickcline for thick horizontal lines of variable length
\newlength\savedwidth
\newcommand\thickcline[1]{%
  \noalign{\global\savedwidth\arrayrulewidth\global\arrayrulewidth 2pt}%
  \cline{#1}%
  \noalign{\vskip\arrayrulewidth}%
  \noalign{\global\arrayrulewidth\savedwidth}%
}

% \thickhline command for thick horizontal lines that span the table
\newcommand\thickhline{\noalign{\global\savedwidth\arrayrulewidth\global\arrayrulewidth 2pt}%
\hline
\noalign{\global\arrayrulewidth\savedwidth}}


% Remove comment for double spacing
%\usepackage{setspace}
%\doublespacing

% Text layout
\raggedright
\setlength{\parindent}{0.5cm}
\textwidth 5.25in
\textheight 8.75in

% Bold the 'Figure #' in the caption and separate it from the title/caption with a period
% Captions will be left justified
\usepackage[aboveskip=1pt,labelfont=bf,labelsep=period,justification=raggedright,singlelinecheck=off]{caption}
\renewcommand{\figurename}{Fig}

% Use the PLoS provided BiBTeX style
% \bibliographystyle{plos2015}

% Remove brackets from numbering in List of References
\makeatletter
\renewcommand{\@biblabel}[1]{\quad#1.}
\makeatother



% Header and Footer with logo
\usepackage{lastpage,fancyhdr,graphicx}
\usepackage{epstopdf}
%\pagestyle{myheadings}
\pagestyle{fancy}
\fancyhf{}
%\setlength{\headheight}{27.023pt}
%\lhead{\includegraphics[width=2.0in]{PLOS-submission.eps}}
\rfoot{\thepage/\pageref{LastPage}}
\renewcommand{\headrulewidth}{0pt}
\renewcommand{\footrule}{\hrule height 2pt \vspace{2mm}}
\fancyheadoffset[L]{2.25in}
\fancyfootoffset[L]{2.25in}
\lfoot{\today}

%% Include all macros below

\newcommand{\lorem}{{\bf LOREM}}
\newcommand{\ipsum}{{\bf IPSUM}}


% tightlist command for lists without linebreak
\providecommand{\tightlist}{%
  \setlength{\itemsep}{0pt}\setlength{\parskip}{0pt}}


% Pandoc citation processing
\newlength{\cslhangindent}
\setlength{\cslhangindent}{1.5em}
\newlength{\csllabelwidth}
\setlength{\csllabelwidth}{3em}
\newlength{\cslentryspacingunit} % times entry-spacing
\setlength{\cslentryspacingunit}{\parskip}
% for Pandoc 2.8 to 2.10.1
\newenvironment{cslreferences}%
  {}%
  {\par}
% For Pandoc 2.11+
\newenvironment{CSLReferences}[2] % #1 hanging-ident, #2 entry spacing
 {% don't indent paragraphs
  \setlength{\parindent}{0pt}
  % turn on hanging indent if param 1 is 1
  \ifodd #1
  \let\oldpar\par
  \def\par{\hangindent=\cslhangindent\oldpar}
  \fi
  % set entry spacing
  \setlength{\parskip}{#2\cslentryspacingunit}
 }%
 {}
\usepackage{calc}
\newcommand{\CSLBlock}[1]{#1\hfill\break}
\newcommand{\CSLLeftMargin}[1]{\parbox[t]{\csllabelwidth}{#1}}
\newcommand{\CSLRightInline}[1]{\parbox[t]{\linewidth - \csllabelwidth}{#1}\break}
\newcommand{\CSLIndent}[1]{\hspace{\cslhangindent}#1}



\usepackage{forarray}
\usepackage{xstring}
\newcommand{\getIndex}[2]{
  \ForEach{,}{\IfEq{#1}{\thislevelitem}{\number\thislevelcount\ExitForEach}{}}{#2}
}

\setcounter{secnumdepth}{0}

\newcommand{\getAff}[1]{
  \getIndex{#1}{}
}

\begin{document}
\vspace*{0.2in}


% Title must be 250 characters or less.
\begin{flushleft}
{\Large
\textbf\newline{How many otolith growth increments are necessary to
reliably estimate growth models and population
age-structure?} % Please use "sentence case" for title and headings (capitalize only the first word in a title (or heading), the first word in a subtitle (or subheading), and any proper nouns).
}
\newline
% Insert author names, affiliations and corresponding author email (do not include titles, positions, or degrees).
\\
Daniel Ricard\textsuperscript{\getAff{Fisheries and Oceans
Canada}}\textsuperscript{*},
Alex Hanke\textsuperscript{\getAff{Fisheries and Oceans
Canada}}\textsuperscript{*},
Lisa Ailloud\textsuperscript{\getAff{Southeast Fisheries Science
Center}},
Cóilín Minto\textsuperscript{\getAff{Galway Mayo Institute of
Technology}},
Olaf Jensen\textsuperscript{\getAff{University of Wisconsin}}\\
\bigskip
\bigskip
* Corresponding author: Daniel.Ricard@dfo-mpo.gc.ca\\
* Corresponding author: Alex.Hanke@dfo-mpo.gc.ca\\
\end{flushleft}
% Please keep the abstract below 300 words
\section*{Abstract}
Lorem ipsum dolor sit amet, consectetur adipiscing elit. Curabitur eget
porta erat. Morbi consectetur est vel gravida pretium. Suspendisse ut
dui eu ante cursus gravida non sed sem. Nullam sapien tellus, commodo id
velit id, eleifend volutpat quam. Phasellus mauris velit, dapibus
finibus elementum vel, pulvinar non tellus. Nunc pellentesque pretium
diam, quis maximus dolor faucibus id. Nunc convallis sodales ante, ut
ullamcorper est egestas vitae. Nam sit amet enim ultrices, ultrices elit
pulvinar, volutpat risus.

% Please keep the Author Summary between 150 and 200 words
% Use first person. PLOS ONE authors please skip this step.
% Author Summary not valid for PLOS ONE submissions.
\section*{Author summary}
Lorem ipsum dolor sit amet, consectetur adipiscing elit. Curabitur eget
porta erat. Morbi consectetur est vel gravida pretium. Suspendisse ut
dui eu ante cursus gravida non sed sem. Nullam sapien tellus, commodo id
velit id, eleifend volutpat quam. Phasellus mauris velit, dapibus
finibus elementum vel, pulvinar non tellus. Nunc pellentesque pretium
diam, quis maximus dolor faucibus id. Nunc convallis sodales ante, ut
ullamcorper est egestas vitae. Nam sit amet enim ultrices, ultrices elit
pulvinar, volutpat risus.

\linenumbers

% Use "Eq" instead of "Equation" for equation citations.
\emph{Text based on plos sample manuscript, see
\url{https://journals.plos.org/ploscompbiol/s/latex}}

\hypertarget{introduction}{%
\section{Introduction}\label{introduction}}

Measuring length of captured individuals is easier and faster than
obtaining detailed information through dissection. Information about
fisheries activities often consists in length frequency samples. A
subset of individuals measured are subjected to more detailed sampling,
including the removal of otoliths.

It is important to determine the appropriate number of individuals to
sample during a survey, in order to ascertain that the goals of the
sampling are fulfilled. In the case of measuring fish and collecting
otoliths under typical sampling protocols, the number of otoliths
collected will scale with the abundance of a species, given that it
follows a wide enough length distribution. The challenge is to then
determine the number of otoliths that have to be aged in order to obtain
reliable estimates of the age-structure of the population of interest.

Under many survey protocols, otoliths are removed from captured
individuals following a length-stratified sampling design. A typical
protocol would be to obtain 2 otoliths for each cm length bin.

Identifying growth annuli on a digital image and obtaining cartesian
coordinates for the annual points along a chosen axis provides length
proxies for each year in the life of an individual. The age-length pairs
derived from growth increments are not independent of each other and are
pseudo-replicates.

A practical question is how many otoliths should be aged in order to
obtain reliable and unbiased estimates of growth and of age structure.

Here are two sample references: {[}1,2{]}.

\hypertarget{methods}{%
\section{Methods}\label{methods}}

\hypertarget{simulation-of-an-age-structured-population}{%
\subsection{Simulation of an age-structured
population}\label{simulation-of-an-age-structured-population}}

An age-structured population dynamics model was used to simulate
observations of fish length and age. The simulation approach generates
observations mimicking those obtained during survey activities and
provide both a reference of the ``true state'' of the population and the
observations from that population.

The goal of the simulated observations is to maintain the uncertainty
known to exist in natural systems and to harness contemporary
computational power to implement robust analyses.

\hypertarget{estimation-of-von-bertalanffy-model-parameters}{%
\subsection{Estimation of von Bertalanffy model
parameters}\label{estimation-of-von-bertalanffy-model-parameters}}

\hypertarget{age-length-keys-and-catch-at-age-matrices}{%
\subsection{Age-length keys and catch-at-age
matrices}\label{age-length-keys-and-catch-at-age-matrices}}

The hybrid forward-inverse age-length key described in {[}3{]} is used
to generate catch-at-age matrices from length samples and age-length
keys.

Catch-at-age matrices used as inputs to age-structured stock assessment
models are used to compute removals in the population and also as tuning
indices for model fitting.

The ``true'' yearly age composition is available from the simulated
age-structured population. The age composition estimated from age-length
keys can be visually compared to the known age composition using
residuals plots. Additionally, a single measure of concordance is the
Relative Mean Square Error (RMSE), which reports the overall agreement
between two catch-at-age matrices.

\hypertarget{growth-increments-of-flatfish-in-the-southern-gulf-of-st.-lawrence}{%
\subsection{Growth increments of flatfish in the southern Gulf of
St.~Lawrence}\label{growth-increments-of-flatfish-in-the-southern-gulf-of-st.-lawrence}}

Digital images of otolihs collected from commercial fishing activities
and scientific trawl surveys in the southern Gulf of St.~Lawrence were
obtained using Leica S9i microscope. The images were sharpened and
enhanced using filters applied using ImageMagick before being uploaded
to the SmartDots software for image annotation. Trained fish agers
annotated the digital images to obtain the annual growth increment
measurements required to perform individual fish growth
back-calculations.

\hypertarget{results}{%
\section{Results}\label{results}}

\hypertarget{simulations}{%
\subsection{Simulations}\label{simulations}}

\hypertarget{american-plaice-otolith-growth-increments}{%
\subsection{American Plaice otolith growth
increments}\label{american-plaice-otolith-growth-increments}}

\hypertarget{discussion}{%
\section{Discussion}\label{discussion}}

\hypertarget{references}{%
\section*{References}\label{references}}
\addcontentsline{toc}{section}{References}

\hypertarget{refs}{}
\begin{CSLReferences}{0}{0}
\leavevmode\vadjust pre{\hypertarget{ref-Feynman1963118}{}}%
\CSLLeftMargin{1. }
\CSLRightInline{Feynman RP, Vernon Jr. FL. The theory of a general
quantum system interacting with a linear dissipative system. Annals of
Physics. 1963;24: 118--173.
doi:\href{https://doi.org/10.1016/0003-4916(63)90068-X}{10.1016/0003-4916(63)90068-X}}

\leavevmode\vadjust pre{\hypertarget{ref-Dirac1953888}{}}%
\CSLLeftMargin{2. }
\CSLRightInline{Dirac PAM. The lorentz transformation and absolute time.
Physica. 1953;19: 888--896.
doi:\href{https://doi.org/10.1016/S0031-8914(53)80099-6}{10.1016/S0031-8914(53)80099-6}}

\leavevmode\vadjust pre{\hypertarget{ref-10.1093ux2ficesjmsux2ffsz072}{}}%
\CSLLeftMargin{3. }
\CSLRightInline{Ailloud LE, Hoenig JM. {A general theory of age-length
keys: combining the forward and inverse keys to estimate age composition
from incomplete data}. ICES Journal of Marine Science. 2019;76:
1515--1523.
doi:\href{https://doi.org/10.1093/icesjms/fsz072}{10.1093/icesjms/fsz072}}

\end{CSLReferences}

\nolinenumbers



\end{document}
