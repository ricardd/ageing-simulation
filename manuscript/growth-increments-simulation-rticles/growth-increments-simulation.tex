\documentclass[preprint, 3p,
authoryear]{elsarticle} %review=doublespace preprint=single 5p=2 column
%%% Begin My package additions %%%%%%%%%%%%%%%%%%%

\usepackage[hyphens]{url}

  \journal{TBD} % Sets Journal name

\usepackage{lineno} % add

\usepackage{graphicx}
%%%%%%%%%%%%%%%% end my additions to header

\usepackage[T1]{fontenc}
\usepackage{lmodern}
\usepackage{amssymb,amsmath}
\usepackage{ifxetex,ifluatex}
\usepackage{fixltx2e} % provides \textsubscript
% use upquote if available, for straight quotes in verbatim environments
\IfFileExists{upquote.sty}{\usepackage{upquote}}{}
\ifnum 0\ifxetex 1\fi\ifluatex 1\fi=0 % if pdftex
  \usepackage[utf8]{inputenc}
\else % if luatex or xelatex
  \usepackage{fontspec}
  \ifxetex
    \usepackage{xltxtra,xunicode}
  \fi
  \defaultfontfeatures{Mapping=tex-text,Scale=MatchLowercase}
  \newcommand{\euro}{€}
\fi
% use microtype if available
\IfFileExists{microtype.sty}{\usepackage{microtype}}{}
\usepackage[]{natbib}
\bibliographystyle{plainnat}

\ifxetex
  \usepackage[setpagesize=false, % page size defined by xetex
              unicode=false, % unicode breaks when used with xetex
              xetex]{hyperref}
\else
  \usepackage[unicode=true]{hyperref}
\fi
\hypersetup{breaklinks=true,
            bookmarks=true,
            pdfauthor={},
            pdftitle={A simulation approach to evaluating the accuracy and precision trade-offs between age-length pairs and growth increments},
            colorlinks=false,
            urlcolor=blue,
            linkcolor=magenta,
            pdfborder={0 0 0}}

\setcounter{secnumdepth}{5}
% Pandoc toggle for numbering sections (defaults to be off)


% tightlist command for lists without linebreak
\providecommand{\tightlist}{%
  \setlength{\itemsep}{0pt}\setlength{\parskip}{0pt}}






\begin{document}


\begin{frontmatter}

  \title{A simulation approach to evaluating the accuracy and precision
trade-offs between age-length pairs and growth increments}
    \author[DFO-Gulf]{Daniel Ricard%
  \corref{cor1}%
  }
   \ead{Daniel.Ricard@dfo-mpo.gc.ca} 
    \author[NOAA-NMFS]{Lisa Ailloud%
  %
  }
   \ead{lisa.ailloud@noaa.gov} 
    \author[DFO-Nfld]{Andrea Perreault%
  %
  }
   \ead{Andrea.Perreault@dfo-mpo.gc.ca} 
      \affiliation[DFO-Gulf]{Gulf Fisheries Centre, Fisheries and Oceans
Canada, Moncton, NB, Canada}
    \affiliation[NOAA-NMFS]{Southeast Fisheries Science Center, Miami
FL, USA}
    \affiliation[DFO-Nfld]{Northwest Atlantic Fisheries Centre,
Fisheries and Oceans Canada, St.~John's, NL, Canada}
    \cortext[cor1]{Corresponding author}
  
  \begin{abstract}
  This is the abstract.
  \end{abstract}
    \begin{keyword}
    keyword1 \sep 
    keyword2
  \end{keyword}
  
 \end{frontmatter}

\hypertarget{introduction}{%
\section{Introduction}\label{introduction}}

Measuring length of captured individuals is easier and faster than
obtaining detailed information through dissection. Information about
fisheries activities often consists in length frequency samples. A
subset of individuals measured are subjected to more detailed sampling,
including the removal of otoliths for age determination.

It is important to determine the appropriate number of individuals to
sample during a survey, in order to ascertain that the goals of the
sampling are fulfilled. In the case of measuring fish and collecting
otoliths under typical sampling protocols, the number of otoliths
collected will scale with the abundance of a species, given that it
follows a wide enough length distribution. The challenge is to then
determine the number of otoliths that have to be aged in order to obtain
reliable estimates of the age-structure of the population of interest.

Under many survey protocols, otoliths are removed from captured
individuals following a length-stratified sampling design. A typical
protocol would be to obtain 2 otoliths for each cm length bin. The
resulting number of otoliths sampled will also change if the efficiency
and size selectivity of the gear change.

Identifying growth annuli on a digital image and obtaining cartesian
coordinates for the annual points along a chosen axis provides length
proxies for each year in the life of an individual. The age-length pairs
derived from growth increments are not independent of each other and are
pseudo-replicates. The ``back-calculation'' of an individual's length
over its lifetime requires that the biological intercept be defined
\citep{Campana-1998-backcalculations}.

A practical arising from the collection of otoliths is how many should
be aged in order to obtain reliable and unbiased estimates of growth and
of age structure.

We use a simulation approach to answer the following questions: 1 - Can
growth individual growth increments be used to compute catch-at-age
matrices?

\hypertarget{methods}{%
\section{Methods}\label{methods}}

\hypertarget{simulation-of-an-age-structured-population}{%
\subsection{Simulation of an age-structured
population}\label{simulation-of-an-age-structured-population}}

The goal of the simulated observations is to maintain the uncertainty
known to exist in natural systems and to harness contemporary
computational power to implement robust analyses.

An age-structured population dynamics model was used to simulate
observations of fish length and age. The simulation approach generates
observations mimicking those obtained during survey activities and
provide both a reference of the ``true state'' of the population and the
observations from that population.

\begin{equation}
R = \bar{R}
\end{equation}

\hypertarget{estimation-of-von-bertalanffy-model-parameters}{%
\subsection{Estimation of von Bertalanffy model
parameters}\label{estimation-of-von-bertalanffy-model-parameters}}

\hypertarget{age-length-keys-and-catch-at-age-matrices}{%
\subsection{Age-length keys and catch-at-age
matrices}\label{age-length-keys-and-catch-at-age-matrices}}

The hybrid forward-inverse age-length key described in
\citet{10.1093/icesjms/fsz072} is used to generate catch-at-age matrices
from length samples and age-length keys.

Catch-at-age matrices used as inputs to age-structured stock assessment
models are used to compute removals in the population and also as tuning
indices for model fitting.

The ``true'' yearly age composition is available from the simulated
age-structured population. The age composition estimated from age-length
keys can be visually compared to the known age composition using
residuals plots. Additionally, a single measure of concordance is the
Relative Mean Square Error (RMSE), which reports the overall agreement
between two catch-at-age matrices.

\hypertarget{results}{%
\section{Results}\label{results}}

\hypertarget{simulations}{%
\subsection{Simulations}\label{simulations}}

\hypertarget{otolith-growth-increments}{%
\subsection{Otolith growth increments}\label{otolith-growth-increments}}

\hypertarget{discussion}{%
\section{Discussion}\label{discussion}}

\renewcommand\refname{References}
\bibliography{mybibfile.bib}


\end{document}
